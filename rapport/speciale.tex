\documentclass{article}

\usepackage{preamble}

\usepackage[style=numeric]{biblatex}
\addbibresource{kasper_speciale_2017.bib}

%\title{Transmembrane protein prediction using long short-term memory networks}
%\author{Kasper Lynderup Jensen}
%\date{\today}

\begin{document}

%\maketitle

\begin{titlepage}
	\centering
	{\scshape\LARGE Aarhus University \par}
	\vspace{1cm}
	{\huge\bfseries Transmembrane Protein Prediction using Long Short-Term Memory Networks\par}
	\vspace{2cm}
	{\Large\itshape Kasper Lynderup Jensen \par}
		20105319\par
	\vfill
	Supervised by\par
	Christian Storm Pedersen
	
	\vfill
	
	% Bottom of the page
	{\large \today\par}
\end{titlepage}

\section*{Abstract}
Transmembrane Protein Prediction is a problem with many uses 
as experimental determination of protein structures is still expensive
and for different purposes it can be useful to know the structure.
Here I introduce a small long short-term memory network based model 
which gives a precision of $67 \pm 3$ and a recall of $71 \pm 3$. 
The model manages, when compared to TMSEG \cite{tmseg}, slightly 
worse but is still a lot better than a simple Hidden Markov Model.

\tableofcontents

% Introducere trans membrane proteiner. hvordan er de representeret, en streng af amino syre
% Forskellige dele af proteinerne har forskellige fordelinger af aminosyre. Dette kan benyttes af en computer
% Introducere trans membrane protein helix prediction, et problem der muligvis kan løses med machine learning
\section{Introduction}
% motivation: Inden for andre fields er nogle af de bedste løsninger med LSTMs. Vi vil se om man kan bruge LSTMs til TMprediction.
% inspiration: TMSEG


% Hvad har andre folk gjort med problemet, nævn nogle stykker og forklar hvad de har gjort
% Slut af med tmseg og beskrev hvad de har gjort mere beskrivende.
% Hvad har andre folk gjort med machine learning problemer baseret på sekvens data
\section{Related work}
\subsection{Transmembrane Protein Prediction}
% "A hidden Markov model for predicting transmembrane helices in protein sequences"\cite{tmhmm} fra bioinformatik-kurset. 
		% At TM prediction er muligt og kan gøres "simpelt" med HMM
% 
% i TMSEG nævnes MEMSAT. måske skal vi skrive noget om den

% Der var en artikel der hedder HTP der virker interressant

% Hvilke modeller eksisterer allerede og bliver brugt pt. 

\subsection{Long Short-Term Memory Networks}
%Find nogle artikler der laver noget nice med lstm

A lot af recent exiting advances in areas where the data is inherently sequence based 
have been done with \gls{lstms}. Areas such as language modelling \cite{JozefowiczEtAl, ShazeerEtAl},
speech recognition \cite{XiongEtAl}, machine translation \cite{WuEtAl} and many more, have had the 
state of the art moved by the use of \gls{lstms}. 



% Beskriv nn. brug nn's problemer til at indroducere rnn
% Beskriv rnn. brug rnn's problemer til at indroducere lstm
% Beskriv lstm
\section{Theory} %\todo{Different name!}
% Beskriv nn. brug nn's problemer til at indroducere rnn
% Beskriv rnn. brug rnn's problemer til at indroducere lstm
% Beskriv lstm

% Hvad er neural network, recurrent nn, lstm
% Tegning/diagram over de forskellige ting - husk at det skal være relevant og skarpt.

\subsection{Neural Network}
% et netværk af perceptrons
	% hver perceptron laver en matematisk udregning.
		% DEN MATEMATISKE UDREGNING
		% feed forward networks bruges bl.a. til image-recognition	
	
\subsection{Recurrent Neural Network}
% \cite{rnnLTD} pointere problemer med rnn

% Recurrent neural network: keyword = recurrent
% kan ikke "huske", hvad der tidligere er set.


\subsection{Long Short-Term Memory Network}
% \cite{lstm} introducere lstm som løsning til ovenfor

% lstm's:
% Indeholder en CellState = LONG short-term memory
% Det er stadig bare en matematisk beregning
% DEN MATEMATISKE BEREGNING
% vi kan benytte det til at kigge på proteinstrenge! yearh!

% Hvordan har vi valgt at prøve at løse problemet.
% Vi har delt problemet op i subproblemer ligesom tmseg
% Vi har prøvet at bruge andre machine learning metoder
% Største forskel er at vi har givet model "rå" data i stedet for feature extraction
% Fordelen er at man ikke behøver at på forhånd finde ud af hvilke features er vigtig

% Hvilke muligheder er der for at implementere modelen. forskellige ml libraries
% Vi har valgt tensorflow
\section{Model}
% Hvordan har vi valgt at prøve at løse problemet.
% Vi har delt problemet op i subproblemer ligesom tmseg
% Vi har prøvet at bruge andre machine learning metoder
% Største forskel er at vi har givet model "rå" data i stedet for feature extraction
% Fordelen er at man ikke behøver at på forhånd finde ud af hvilke features er vigtig

% Hvilke muligheder er der for at implementere modelen. forskellige ml libraries
% Vi har valgt tensorflow

% Noget om hvorfor lstm
	% Der er flere beregninger når man bruger lstms (con)
	% Long short-term memory: Hvorfor har vi brug for LONG memory?
% Hvad gør vores model forskelligt fra tmseg

% Hvad forventer vi vores model kan gøre bedre end de modeller der allerede eksisterer?

% tegning af model!!

%  Problem delt op i 4 steps som i tmseg
% Kun 3 af stepsne er machine learning
	% Første step assigner en sandsynlighed for hver klasse til hver position i sekvens
		% Vi har gjort det med en lstm, fodret med den rå sekvens
		% Første lag embeder sekvens data i et vector rum
		% Andet lag er en bidirectionel lstm 
		% Trejde og sidste lag er et softmax layer som giver en sansynligheds fordeling af de forskellige klasser
		
	% Andet step er at give hver position en klasse ud fra dens sandsynlighed 
		% med forbehold for at helixer ikke må være for korte
		% Måske tilføje et bias til nogle af klasserne
		% Ikke machine learning men mere post processing af af første step
		
	% Trejde step er justering af enderne af helixerne
	
	% Fjerne step er finde en topologi af sekvens, 
		% ie hvilken vej vender helixerne og hvilke dele er inden for membranen og hvilke er uden for

In many \gls{ann} based models it is very hard to ensure some syntax rules for the output. If the output 
classes was \emph{\gls{tmh}}, \emph{inside} and \emph{outside}, then it should not be possible for a output sequence to contain 
adjacent positions with one being inside and the other being outside or both ends of a \gls{tmh} to be on the 
same side, but these rules cannot be ensured. If the problem is divided in sub problems in such a way that 
the rules is in the way the problem is divided. This is what was done in TMSEG\cite{tmseg}, they divided 
the problem such that each sub-problem is more focused on one thing and most of the desired structure is 
enforced in the design of the problem. They still do some post processing to constrain the output in certain ways, 
signal peptides is only in the beginning and \glspl{tmh} is not too short, but this is much simpler than 
somehow constraining the order the classes is allowed to appear in. The other advantages of the sub-division 
is the ability of each sub-problem to focus on one thing. In TMSEG the problem was divided into four steps
where the focus of the first step was to identify regions of the protein with \glspl{tmh} or signal peptides.
The second step's focus was post processing of the first step, to remove noise and constrain the output.
The focus of the third step was to adjust the endpoints of the \glspl{tmh} to give a more precise location.
The fourth and final step's focus was to assign a topology to the protein. 

To make this model I have chosen to use the same division of the problem as in TMSEG because of the 
advantages listed above. I chosen to use different machine learning methods to examine the feasibility 
of giving the model the raw data and letting it learn what is important and to look into which trade-offs
there is to doing it this way instead of using expert knowledge about the problem to choosing and extracting 
the important features. \glspl{lstm} is chosen as the main machine learning method because it is very suitable
to use on raw sequence data and because of good results in a lot of different problems. 

Ideally I would have looked at all steps and compared them individually with the corresponding step in TMSEG,
but due to time constraint I have concentrated on the first three steps and mainly compared them together.

\begin{figure}
	\centering
%	\includegraphics[width=\textwidth]{}
	\caption{The layers for the model of step 1}
	\label{fig:step1}
\end{figure}


\subsection{Machine Learning Libraries}

\subsection{Implementation}
% Hvorfor tensorflow, python
% Hvad får vi fra tensorflow, hvilket arbejde har vi selv gjort
% Hvor meget arbejde ligger der i de forskellige dele af model

% Tensorflow giver selve machine learning bygge klodserne 
% men vi har selv sat dem sammen til en model 
% For og efter behandling af dataen har der også lagt meget arbejde i

% Beskriv alle experimenter vi har kørt
\section{Experiments}
% which experiements will we do on our model?
	% List them - gør det overskueligt!!!
	
% Hvad vil det sige at predicte en helix
% Skal hver position have den rigtige klasse
% Skal den ses som et samlet strykke af positioner med samme klasse.
% i så fald hvornår er den rigtig.
% Vi har brugt to forskellige mål for hvornår en prediction er rigtig
% Den første kræver 50% overlap og ender inden for 5 position for at den gælder som rigtig
% Den anden kræver kun 25% overlap

	
% The results we allready have of the other model - or the tests we need to do, to be able to compare the models.
% the data used

% Hvad forventer vi resultaterne vil være.

All the weights and biases in the model gets initialized by random values and the training
is done by mini-batch gradient descent, where a random batch of the samples is used to 
calculate the gradient. The training can converge to different local minima with 
different initial variables and order of batches.
The trained model is therefore very sensitive to the initial assignment of 
variables and the order the samples is used. It is therefore unlikely to get the same 
result every time the model is trained and can have a very high variance. 
To try to get an idea about the variance I will run each experiment multiple times.


\subsection{Dataset}
The dataset used to train and test the model is the TMP166\cite{tmseg} dataset used to 
develop TMSEG. The dataset consists of four subsets with approximate the same distribution 
of different length of proteins. The fourth subset was omitted from the training and only 
used to test the final model on. This was done to be sure that the model was not overfitted
on the test set and that the model was generalizing to new data. 

\subsection{Training}
\glspl{lstm} is fairly slow in terms of time, both to inference and especially to train.
This is because \gls{lstm} layers needs lots of computations in comparison with ordinary 
fully connected layers and most \glspl{ann} is already slow in comparison with many 
other types of machine learning.
\todo{maybe some timing test and comparison with HMM}

Proper grid search of hyper-parameters is a very long process because the time it takes
scales exponential in the number of different hyper-parameters and the model has a lot 
of them and together with the fairly long training time is the whole process very slow.
I have therefore chosen not to do a complete grid search.
\todo{beskriv hyper-parameter valgs process}


\subsection{Measurements}
To evaluate the model some measure for the performance of the model have to be chosen.
I have chosen three different types of measurements that have three different purposes.
The loss function used in training of step 1 assigns a loss to each position with a 
wrong predicted class and the optimiser then tries to minimize this loss. 
Since the loss is calculated from each position, it seems apparent to also use a 
position based measurement. Especially to measure the performance of step 1.
This type of measurement does not see a \gls{tmh} as a single thing and can therefore
not say anything about the number of \glspl{tmh}. It is therefore also interesting 
to have a measure for this. I have used two different types of measurements to do this, 
both of them counts a consecutive sequence of positions classified as \gls{tmh} as a single
prediction. The purpose of step 1 was to identify regions with \glspl{tmh}.
The second measurement type's purpose is therefore to measure if a prediction of a 
\gls{tmh} is in the right region, this is done by finding the length of the overlap 
between the predicted \gls{tmh} and the true \gls{tmh}, the length is then compared 
to the longest of the two \gls{tmh}. The predicted \gls{tmh} is counted as a correct prediction
if the ratio is larger than a certain value. I have done this with different values 
to see how precise the predicted regions is. 
The last measurement type is intended to measure the usefulness of step 3. The purpose of
this step was to adjust the endpoints of predicted \glspl{tmh}, this measurement 
therefore looks at endpoints and counts a predicted \gls{tmh} as correct if both endpoints 
is a certain number of positions of the true endpoints. This is again done with 
different values for the allowed distance to the true endpoints. 
These two measurement types is derived from one of the measurement they use in TMSEG,
which is a combination of the two where the overlap have to be at least 50\% and the 
endpoints have to be within 5 positions of the true endpoints for a prediction to be counted
as correct. I have also used this measurement to be able to compare to the result they get 
there. 




% Sammenlign resultater mellem experimenterne og mellem andre modeller
% Kan vi se noget interressant i hvordan weights ser ud? 


\section{Evaluation of the model}

\subsection{Results and evaluation of the results}
% the results with nice grafic representation
% sub-conclusion on each little test where we look at the results in TMSEG

\subsection{Discussion}
% What have we achieved? 
% Further work for it to be even better. 


\section{Conclusion}
% gav det så mening at bruge LSTMs?

I have build small \gls{lstm} based model which achieved a precision of 
$67 \pm 3$ and a recall of $71 \pm 3$ from only raw data as input.
This shows that it is feasible to get a decent result from a 
\gls{lstm} based model without the use of feature extraction.
Although the result was not as impressive as TMSEG it still shows 
potential for \gls{lstm} based models. It does however show that
training time could become a problem if the model gets larger and more 
complex. 

\printbibliography

\end{document}}