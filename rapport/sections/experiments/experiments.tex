\section{Experiments}
% which experiements will we do on our model?
	% List them - gør det overskueligt!!!
	
% Hvad vil det sige at predicte en helix
% Skal hver position have den rigtige klasse
% Skal den ses som et samlet strykke af positioner med samme klasse.
% i så fald hvornår er den rigtig.
% Vi har brugt to forskellige mål for hvornår en prediction er rigtig
% Den første kræver 50% overlap og ender inden for 5 position for at den gælder som rigtig
% Den anden kræver kun 25% overlap

	
% The results we allready have of the other model - or the tests we need to do, to be able to compare the models.
% the data used

% Hvad forventer vi resultaterne vil være.

All the weights and biases in the model gets initialized by random values and the training
is done by mini-batch gradient descent, where a random batch of the samples is used to 
calculate the gradient. The training can converge to different local minima with 
different initial variables and order of batches.
The trained model is therefore very sensitive to the initial assignment of 
variables and the order the samples is used. It is therefore unlikely to get the same 
result every time the model is trained and can have a very high variance. 
To try to get an idea about the variance I will run each experiment multiple times.


\subsection{Dataset}
The dataset used to train and test the model is the TMP166\cite{tmseg} dataset used to 
develop TMSEG. The dataset consists of four subsets with approximate the same distribution 
of different length of proteins. The fourth subset was omitted from the training and only 
used to test the final model on. This was done to be sure that the model was not overfitted
on the test set and that the model was generalizing to new data. 

\subsection{Training}
\glspl{lstm} is fairly slow in terms of time, both to inference and especially to train.
This is because \gls{lstm} layers needs lots of computations in comparison with ordinary 
fully connected layers and most \glspl{ann} is already slow in comparison with many 
other types of machine learning.
\todo{maybe some timing test and comparison with HMM}

Proper grid search of hyper-parameters is a very long process because the time it takes
scales exponential in the number of different hyper-parameters and the model has a lot 
of them and together with the fairly long training time is the whole process very slow.
I have therefore chosen not to do a complete grid search.
\todo{beskriv hyper-parameter valgs process}


\subsection{Measurements}
To evaluate the model some measure for the performance of the model have to be chosen.
I have chosen three different types of measurements that have three different purposes.
The loss function used in training of step 1 assigns a loss to each position with a 
wrong predicted class and the optimiser then tries to minimize this loss. 
Since the loss is calculated from each position, it seems apparent to also use a 
position based measurement. Especially to measure the performance of step 1.
This type of measurement does not see a \gls{tmh} as a single thing and can therefore
not say anything about the number of \glspl{tmh}. It is therefore also interesting 
to have a measure for this. I have used two different types of measurements to do this, 
both of them counts a consecutive sequence of positions classified as \gls{tmh} as a single
prediction. The purpose of step 1 was to identify regions with \glspl{tmh}.
The second measurement type's purpose is therefore to measure if a prediction of a 
\gls{tmh} is in the right region, this is done by finding the length of the overlap 
between the predicted \gls{tmh} and the true \gls{tmh}, the length is then compared 
to the longest of the two \gls{tmh}. The predicted \gls{tmh} is counted as a correct prediction
if the ratio is larger than a certain value. I have done this with different values 
to see how precise the predicted regions is. 
The last measurement type is intended to measure the usefulness of step 3. The purpose of
this step was to adjust the endpoints of predicted \glspl{tmh}, this measurement 
therefore looks at endpoints and counts a predicted \gls{tmh} as correct if both endpoints 
is a certain number of positions of the true endpoints. This is again done with 
different values for the allowed distance to the true endpoints. 
These two measurement types is derived from one of the measurement they use in TMSEG,
which is a combination of the two where the overlap have to be at least 50\% and the 
endpoints have to be within 5 positions of the true endpoints for a prediction to be counted
as correct. I have also used this measurement to be able to compare to the result they get 
there. 


