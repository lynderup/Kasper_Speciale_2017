\section{Related work}
% Hvad har andre folk gjort med problemet, nævn nogle stykker og forklar hvad de har gjort
% Slut af med tmseg og beskrev hvad de har gjort mere beskrivende.
% Hvad har andre folk gjort med machine learning problemer baseret på sekvens data

\subsection{Transmembrane Protein Prediction}
% "A hidden Markov model for predicting transmembrane helices in protein sequences"\cite{tmhmm} fra bioinformatik-kurset. 
		% At TM prediction er muligt og kan gøres "simpelt" med HMM
% 
% i TMSEG nævnes MEMSAT. måske skal vi skrive noget om den

% Der var en artikel der hedder HTP der virker interressant

% Hvilke modeller eksisterer allerede og bliver brugt pt. 

\subsubsection{TMSEG}


\subsection{Long Short-Term Memory Network}
%Find nogle artikler der laver noget nice med lstm

A lot af recent exiting advances in areas where the data is inherently sequence based 
have been done with \glspl{lstm}. Areas such as language modelling \cite{JozefowiczEtAl, ShazeerEtAl},
speech recognition \cite{XiongEtAl}, machine translation \cite{WuEtAl} and many more, have had the 
state of the art moved by the use of \glspl{lstm}. 

% Åbenbart bruger de slet ikke lstm til deres branching eller expert layers. Det virker ikke til de bruger lstm'er på en ny måde
%In \cite{ShazeerEtAl} a \gls{lstm} is used as a gating network to enable conditional computations of a 
%much larger network. 

In \cite{JozefowiczEtAl} they explore different model architectures for language modelling 
on very large datasets, and comes to the conclusion that \glspl{lstm} is the best of the models
they tried and the bigger layers they used, the better results they got. Most of their work is
in decreasing the number of computations necessary to train such large \glspl{lstm}. 
Amongst other they used a approximation of the softmax layer. Due to the large number of 
classes in langauge modelling the softmax layer becomes very large and have a large influence 
on the number of computations. 