\section{Introduction}
% Introducere trans membrane proteiner. hvordan er de representeret, en streng af amino syre
% Forskellige dele af proteinerne har forskellige fordelinger af aminosyre. Dette kan benyttes af en computer
% Introducere trans membrane protein helix prediction, et problem der muligvis kan løses med machine learning

% motivation: Inden for andre fields er nogle af de bedste løsninger med LSTMs. Vi vil se om man kan bruge LSTMs til TMprediction.
% inspiration: TMSEG

\glspl{tmp} is a kind of proteins that crosses the membrane, they have an important
role in many living organisms \cite{} and different areas is interested in theirs structure,
among other as drug targets \cite{}, but it can be expensive to experimental determine 
their structure, and it could therefore be beneficial if it was possible to determine
the structure from the protein's composition. This is the problem of \gls{tmp} prediction.
Instances in this problem comes as a string of characters corresponding to the sequence 
of amino acids the protein consists of. The goal is then to predict the structure 
from this sequence. The structure consists of three parts, the parts of the protein 
that goes through the membrane, called \glspl{tmh}, the part that is inside, and the part 
that is outside. The structure is represented as a string the same length as the 
amino acid sequence and consists of different characters corresponding to which part
of the structure, the amino acid at the same position belong to.
This is possible because the distribution of amino acids is different in the different 
parts. This distribution or pattern can be approximated by training a machine learning 
model. The training is done by showing the model examples of \glspl{tmp} with known
structure.

Here I will build a machine learning model to try to solve this problem. The model
will be based on the model TMSEG \cite{tmseg}, it will use different machine learning
methods but will follow TMSEG in both how the problem is approached and how the result
is measured. Most traditional machine learning approaches needs a lot of knowledge 
about the problem to be able to extract the relevant information from the input data.
Here i will examine the possibility of skipping this step by feeding the raw input 
into the model, this I have done by basing the model on a type of \glspl{rnn} called 
\glspl{lstm}. \glspl{lstm} have had a lot of success in other areas and here I will 
examine their suitability in this problem. I will run a number of experiments to determine
how this model compares with other models. 