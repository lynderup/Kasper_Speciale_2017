\section{Evaluation}

\subsection{Results}


\begin{table}
	\centering 
	\begin{tabular}{l|c|c}
		Model & Precision & Recall \\ \hline
		Step1 w/o PSSMs & $79 \pm 2$ & $84 \pm 2$ \\
		Step1 w PSSMs   & $82 \pm 0$ & $85 \pm 1$ \\
		Step2 w/o PSSMs & $80 \pm 2$ & $84 \pm 2$ \\
		Step2 w PSSMs   & $82 \pm 1$ & $85 \pm 1$ \\
		Step3 w/o PSSMs & $78 \pm 4$ & $80 \pm 7$ \\
		Step3 w PSSMs   & $81 \pm 4$ & $70 \pm 8$ \\
		HMM & $72 \pm 0$ & $31 \pm 0$ \\
	\end{tabular}
	\caption{Precision and recall for the position based measurement.}
	\label{tab:char}
\end{table}

The precision and recall for the position based measurement is shown in 
Table \ref{tab:char}. The model gives a moderately improvement to precision
in comparison with the \gls{hmm} and a massively improvement to recall.
Step 2 and 3 does almost nothing for the result. This is because the changes
they do is only a tiny part of all positions and therefore have little to 
no impact on the result of this measurement. The use of \glspl{pssm} 
gives a little better results and a little less standard deviation. 

\begin{table}
	\centering 
	\begin{subtable}[]{\textwidth}
		\begin{tabular}{l|c|c|c|c|c|c|c|c} 
			Model & \multicolumn{2}{|c}{10\%}& \multicolumn{2}{|c}{25\%}& \multicolumn{2}{|c}{50\%}& \multicolumn{2}{|c}{75\%}\\ 
			& Precis & Recall & Precis & Recall & Precis & Recall & Precis & Recall \\ \hline 
			Step 1  & $ 89 \pm 2 $ & $ 90 \pm 3 $ & $ 87 \pm 2 $ & $ 88 \pm 3 $ & $ 80 \pm 2 $ & $ 81 \pm 4 $ & $ 59 \pm 4 $ & $ 60 \pm 5 $ \\
			w PSSMs & $ 88 \pm 3 $ & $ 92 \pm 1 $ & $ 87 \pm 3 $ & $ 91 \pm 1 $ & $ 80 \pm 3 $ & $ 84 \pm 2 $ & $ 59 \pm 2 $ & $ 62 \pm 3 $ \\
			Step 2  & $ 93 \pm 1 $ & $ 94 \pm 3 $ & $ 92 \pm 1 $ & $ 92 \pm 2 $ & $ 86 \pm 2 $ & $ 87 \pm 3 $ & $ 63 \pm 5 $ & $ 64 \pm 5 $ \\
			w PSSMs & $ 92 \pm 1 $ & $ 98 \pm 2 $ & $ 91 \pm 0 $ & $ 97 \pm 1 $ & $ 86 \pm 1 $ & $ 92 \pm 2 $ & $ 64 \pm 4 $ & $ 68 \pm 4 $ \\
			Step 3  & $ 93 \pm 1 $ & $ 92 \pm 5 $ & $ 91 \pm 1 $ & $ 90 \pm 5 $ & $ 83 \pm 4 $ & $ 82 \pm 7 $ & $ 51 \pm 14$ & $ 51 \pm 16$ \\
			w PSSMs& $ 92 \pm 1 $ & $ 95 \pm 4 $ & $ 89 \pm 2 $ & $ 93 \pm 3 $ & $ 75 \pm 8 $ & $ 77 \pm 7 $ & $ 28 \pm 14$ & $ 28 \pm 13$ \\
		\end{tabular}
		\caption{Precision and recall for 10\%, 25\%, 50\% and 75\% overlap of true \gls{tmh} and predicted \gls{tmh}.}
		\label{tab:overlap}
	\end{subtable}

	\begin{subtable}[]{\textwidth}
		\begin{tabular}{l|c|c|c|c|c|c|c|c} 
			Model & \multicolumn{2}{|c}{Below 10}& \multicolumn{2}{|c}{Below 5}& \multicolumn{2}{|c}{Below 2}& \multicolumn{2}{|c}{Equal 0}\\ 
			& Precis & Recall & Precis & Recall & Precis & Recall & Precis & Recall \\ \hline 
			Step 1  & $ 78 \pm 3 $ & $ 79 \pm 4 $ & $ 60 \pm 4 $ & $ 61 \pm 5 $ & $ 24 \pm 2 $ & $ 24 \pm 2 $ & $  2 \pm 1 $ & $  2 \pm 1 $ \\
			w PSSMs & $ 78 \pm 2 $ & $ 82 \pm 1 $ & $ 62 \pm 1 $ & $ 65 \pm 2 $ & $ 26 \pm 2 $ & $ 28 \pm 2 $ & $  1 \pm 1 $ & $  1 \pm 1 $ \\
			Step 2  & $ 85 \pm 3 $ & $ 85 \pm 3 $ & $ 65 \pm 5 $ & $ 65 \pm 5 $ & $ 25 \pm 2 $ & $ 26 \pm 2 $ & $  2 \pm 1 $ & $  2 \pm 1 $ \\
			w PSSMs & $ 85 \pm 1 $ & $ 90 \pm 1 $ & $ 67 \pm 3 $ & $ 71 \pm 3 $ & $ 29 \pm 3 $ & $ 30 \pm 3 $ & $  1 \pm 1 $ & $  1 \pm 1 $ \\
			Step 3  & $ 82 \pm 3 $ & $ 81 \pm 6 $ & $ 54 \pm 11$ & $ 54 \pm 13$ & $ 19 \pm 9 $ & $ 19 \pm 9 $ & $  1 \pm 1 $ & $  1 \pm 1 $ \\
			w PSSMs & $ 77 \pm 3 $ & $ 80 \pm 4 $ & $ 39 \pm 10$ & $ 40 \pm 9 $ & $ 10 \pm 7 $ & $ 10 \pm 7 $ & $  0 \pm 0 $ & $  0 \pm 0 $ \\
		\end{tabular}
		\caption{Precision and recall for endpoints difference between true 
			\gls{tmh} and predicted \gls{tmh} below or equal to 10, 5, 2 and 0.}
		\label{tab:endpoint}
	\end{subtable}
	\caption{Results of different overlap ratios and endpoints differences comparing the different layers of the model.
	Error rates is specified as the standard deviation of the multiple runs of the experiments.}
	\label{tab:step_compare}
\end{table}

Table \ref{tab:step_compare} contains the result for different measurements 
comparing the steps in the model. Unsurprising goes both the precision and the 
recall down when the requirement for the predictions gets stricter. In 
Table \ref{tab:overlap} can it be seen that the difference between $10\%$ and $25\%$
overlap is very small and within one standard deviation from each other, 
this is implying to that if the model have made a prediction it has more than 
$25\%$ or less than $10\%$ overlap with a true \gls{tmh}.
Table \ref{tab:endpoint} shows that the changing the endpoint requirement a little 
gives a big difference in the result, from a precision on $85 \pm 1$ for an endpoints 
difference up to $10$ to a precision on $1 \pm 1$ for no endpoints difference.
Both precision and recall is very low if the requirement for a predicted \gls{tmh}
to be correct is that there is no difference in the endpoints, ie. the prediction 
is $100\%$ correct, which means there are almost no $100\%$ correct predictions.
From both measurement types can it be seen that step 2 helps a lot on especial 
the precision. This is because it removes a lot of small \glspl{tmh} from the 
prediction that does not correspond to real \glspl{tmh}, these small predicted 
\glspl{tmh} was lowering the precision but don't have much impact on the recall.
Here it also seems to give a little better results by using \glspl{pssm}
for step 1 and consequently step2 and also lowers the standard deviation a little,
But for step 3 it gives worse results to use \glspl{pssm}, but still mostly lowers 
the standard deviation a little. For $75\%$ overlap and endpoints difference below 
2 it even halves the result, but these have a high standard deviation.
The results from step 3 is in general not very good, it doesn't change the result for 
$10\%$ and $25\%$ overlap much, but for $50\%$ overlap and endpoints difference 
below 10 it get slightly worse than step 2 and for $75\%$ overlap and endpoints 
difference below 5 it considerable lowers both precision and recall
and for endpoints difference below 
2, step 3 almost halves the results. The increasing impact step 3 has on the 
result as the requirement for the predictions gets stricter, is because step 3 
can only move the endpoints 6 positions and 6 positions don't have much impact 
if a prediction only has to have $10\%$ overlap but have a very large impact 
if the endpoint difference has to be below 2. 
Looking closer at the prediction from step 3 used to adjust the endpoints, it 
seems to give the same probability for every segment. The probabilities only
differs in the fifth decimal point and looks more like rounding inaccuracy
than predictions. The model does not seem to be able to learn any meaningful
thing from the input and have minimized the loss by always predicting the 
distribution of the labels in the training data. 
This failure to learn the patterns in the input data to accurately predict
the start end end of \glspl{tmh} could be because patterns is too complex or 
indistinct to learn with the size of the dataset or with the layout of the 
model.

\begin{table}
	\centering 
	\begin{tabular}{l|c|c} 
		Model & Precision & Recall \\ \hline
		Step 1 & $60 \pm 4$ & $61 \pm 5$ \\
		w PSSMs& $62 \pm 1$ & $65 \pm 2$ \\
		Step 2 & $65 \pm 5$ & $65 \pm 5$ \\
		w PSSMs& $67 \pm 3$ & $71 \pm 3$ \\
		Step 3 & $54 \pm 11$ & $54 \pm 13$ \\
		w PSSMs& $39 \pm 10$ & $40 \pm 9$ \\
		HMM   & $34 \pm 0$ & $13 \pm 0$ \\ 
		TMSEG\cite{tmseg} & $87 \pm 3$ & $84 \pm 3$
	\end{tabular}
    \caption{Precision and recall for 50\% overlap and endpoints difference below 5.}
	\label{tab:pr50}
\end{table}

In Table \ref{tab:pr50} is the precision and recall from using the same 
measurement as used by TMSEG. It also shows how the model compares with 
other models. TMSEG is a lot better at both precision and recall, 
but the model is a big improvement to the simple \gls{hmm}
based model. As before gives the use af step 3 a worse result than stopping
after step 2 and the best result is from step 2 where \glspl{pssm} was 
used, where it gives a precision of $67 \pm 3$ and a recall of $71 \pm 3$.

\subsection{Running time}
\label{sec:time}

%Inference time for step 1: 1.6142690181732178
%Inference time for step 2: 0.0045070648193359375
%Inference time for step 3: 2.752535581588745
%Training time for step 1: 800.2764098644257
%Training time for step 3: 1267.4366292953491

%Training time: 0.049510955810546875
%Decoding time: 0.8011248111724854

\begin{table}
	\centering 
	\begin{tabular}{l|c|c} 
		Model & Training & Inference \\ \hline
		Step 1 & 800 & 1.61 \\ 
		Step 2 & - & 0.005 \\ 
		Step 3 & 1267 & 2.75 \\
		HMM   & 0.0495 & 0.80
	\end{tabular}
	\caption{Training and inference timings for the different steps 
		and for a \gls{hmm}. All numbers in seconds. Training was done
		on the three first subset and inference was done on the last.}
	\label{tab:time}
\end{table}

Table \ref{tab:time} contains timings of the training and inference of each 
step of the model and of the simple \gls{hmm}. It takes a lot longer to train 
step 1 and step 3 than it does to use them to inference. 
It takes a lot longer to use the \gls{hmm} for inference than it does to train
it but it is still faster than the \glspl{lstm}. Since training only have to be
done once and the trained model can be saved and used later, the slow training 
time is not a huge disadvantages in comparison with other models. The inference
time is more important, because that is the time that is relevant if the model 
was to be used, and while it is also slower than the \gls{hmm} for inference,
it is not much slower.

\subsection{Discussion}
% What have we achieved? 
% Further work for it to be even better. 
%\subsection{Learning what the Model has Learned}
% Can we learn something from what the model has learn. By somehow visualising the weight matrices we can
% se what the model has found to be importen and sometimes something interresting aboubt the problem can be learned
Tensorflow was very easy, after getting used to build a computation graph and 
then running it, to use to make the implementations of both machine learning models,
but was very bad at other part of the model because it was very cumbersome 
to make loops, branches and other normal control flow computations.
It works very good at linear algebra as it was designed to but not 
really at any thing else and is therefore a tool that only should be used
at the thing it was designed to and not every problem.

The layout of the model used here is almost the simplest that is needed to
get a model based on a \gls{lstm} to work. It is very small, 50 units in each 
directions in the bidirectional layer and 100 units in the forward layer 
on top of that, compared to the 8192 units used for each \gls{lstm} layer 
used in \cite{JozefowiczEtAl}. The layer was also shallow with only
two \gls{lstm} layers compared to 16 layers used in \cite{WuEtAl}.
Despite the small size of the model it gives fairly good results,
a lot better than a simple \gls{hmm}, but not as good as TMSEG,
so it may be possible to get results comparable to TMSEG with a bigger and 
more complicated \gls{lstm} based model. Even this small model uses a lot of 
time for training as shown in Section \ref{sec:time} and the training time
gets a lot bigger with a bigger network. For the 16 layers model used in 
\cite{WuEtAl} they use 96 GPUs for 6 days to train, so the large models is 
only feasible to train with access to clusters of specialized hardware,
but once trained they are much faster on inference on new data.
To take advantages of large model, large datasets is an advantages
and the datasets with annotated \glspl{tmp} are not that large, 
so it might not even give better results using a larger network 
with the dataset used here. There exists techniques to make a dataset larger
without making new observation by copying existing examples and introducing 
errors and noise that is similar to what might exists in real examples,
this also have the advantages to make the model more robust against noise 
in the data. 

Even with all the data as used by TMSEG accessible, the model doesn't
seem to be able to learn as much from the raw data as TMSEG is able
to learn from the carefully extracted features, this could be because
the model is not large enough or the dataset large enough to be able to 
learn the more concealed patterns that is more easily learned from extracted
features, but it could also be that a \gls{lstm} is simply not able to learn 
them although I find that unlikely. I definitely find that the result found
here shows potential for \gls{lstm} based model for \gls{tmp} prediction.

\subsection{Further Work}
The model does not predict the topology of the structure, i.e. which 
non-\gls{tmh} parts are inside the membrane and which parts are outside.
Step 4 of TMSEG is the part of their model that does this and if I had more
time was this the first improvement I would add to this model.
Other interesting things to take a look at is other classes in the 
structure like signal peptides and reentrant loops. 
The model was also only trained to look at \glspl{tmp} and not 
soluble proteins and can therefore not be used to discriminate 
between those. This is something TMSEG does really well and 
something I would have looked at if I had more time.
For a more distant goal it could be interesting to examine other \gls{rnn}
architectures to see if others was more suitable to the problem and
if it was possible to improve the result by also using feature extraction.



