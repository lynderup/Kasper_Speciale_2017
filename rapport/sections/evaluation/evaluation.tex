\section{Evaluation}

\subsection{Results}

\begin{table}
	\centering 
	\begin{subtable}[]{\textwidth}
		\begin{tabular}{l|c|c|c|c|c|c|c|c} 
			Model & \multicolumn{2}{|c}{10\%}& \multicolumn{2}{|c}{25\%}& \multicolumn{2}{|c}{50\%}& \multicolumn{2}{|c}{75\%}\\ 
			& Precis & Recall & Precis & Recall & Precis & Recall & Precis & Recall \\ \hline 
			Step1 & $ 89 \pm 1 $ & $ 92 \pm 2 $ & $ 88 \pm 2 $ & $ 90 \pm 2 $ & $ 81 \pm 3 $ & $ 84 \pm 3 $ & $ 62 \pm 3 $ & $ 63 \pm 4 $ \\ 
			Step2 & $ 92 \pm 1 $ & $ 94 \pm 2 $ & $ 92 \pm 1 $ & $ 94 \pm 2 $ & $ 85 \pm 2 $ & $ 87 \pm 2 $ & $ 64 \pm 3 $ & $ 65 \pm 3 $ \\ 
			Step3 & $ 92 \pm 1 $ & $ 92 \pm 2 $ & $ 91 \pm 1 $ & $ 91 \pm 2 $ & $ 80 \pm 2 $ & $ 80 \pm 3 $ & $ 42 \pm 4 $ & $ 42 \pm 4 $ \\ 
		\end{tabular}
		\caption{Precision and recall for 10\%, 25\%, 50\% and 75\% overlap of true \gls{tmh} and predicted \gls{tmh}.}
		\label{tab:overlap}
	\end{subtable}
	
	\begin{subtable}[]{\textwidth}
		\begin{tabular}{l|c|c|c|c|c|c|c|c} 
			Model & \multicolumn{2}{|c}{Below 10}& \multicolumn{2}{|c}{Below 5}& \multicolumn{2}{|c}{Below 2}& \multicolumn{2}{|c}{Equal 0}\\ 
			& Precis & Recall & Precis & Recall & Precis & Recall & Precis & Recall \\ \hline 
			Step1 & $ 80 \pm 3 $ & $ 82 \pm 3 $ & $ 64 \pm 3 $ & $ 66 \pm 3 $ & $ 26 \pm 3 $ & $ 27 \pm 4 $ & $  2 \pm 1 $ & $  2 \pm 1 $ \\ 
			Step2 & $ 84 \pm 2 $ & $ 86 \pm 2 $ & $ 67 \pm 3 $ & $ 69 \pm 3 $ & $ 27 \pm 3 $ & $ 28 \pm 3 $ & $  2 \pm 1 $ & $  2 \pm 1 $ \\ 
			Step3 & $ 77 \pm 2 $ & $ 77 \pm 3 $ & $ 49 \pm 3 $ & $ 49 \pm 4 $ & $ 18 \pm 2 $ & $ 18 \pm 2 $ & $  1 \pm 1 $ & $  1 \pm 1 $ \\ 
		\end{tabular}
		\caption{Precision and recall for endpoints difference between true 
			\gls{tmh} and predicted \gls{tmh} below or equal to 10, 5, 2 and 0.}
		\label{tab:endpoint}
	\end{subtable}
	\caption{Results of different overlap ratios and endpoints differences comparing the different layers of the model.
	Error rates is specified as the standard deviation of the multiple runs of the experiments.}
	\label{tab:step_compare}
\end{table}

Table \ref{tab:step_compare} contains the result for different measurements 
comparing the steps in the model. Unsurprising goes both the precision and the 
recall down when the requirement for the predictions gets stronger. In 
Table \ref{tab:overlap} can it be seen that the difference between $10\%$ and $25\%$
overlap is very small and within one standard deviation from each other, 
this is implying to that if the model have made a prediction it has more than 
$25\%$ or less than $10\%$ overlap with a true \gls{tmh}.
Table \ref{tab:endpoint} shows that the changing the endpoint requirement a little 
gives a big difference in the result, from a precision on $84 \pm 2$ for an endpoints 
difference up to $10$ to a precision on $2 \pm 1$ for no endpoints difference.
Both precision and recall is very low if the requirement for a predicted \gls{tmh}
to be correct is that there is no difference in the endpoints, ie. the prediction 
is $100\%$ correct, which means there are almost no $100\%$ correct predictions.
From both measurement types can it be seen that step 2 helps a lot on especial 
the precision. This is because it removes a lot of small \glspl{tmh} from the 
prediction that does not correspond to real \glspl{tmh}, these small predicted 
\glspl{tmh} was lowering the precision but don't have much impact on the recall.
The results from step 3 is not very good, it doesn't change the result for 
$10\%$ and $25\%$ overlap, but it is still slightly worse than step 2. 
For $50\%$ overlap and endpoints difference below 5 and 10, it considerable lowers
both precision and recall and for $75\%$ overlap and endpoints difference below 
2, step 3 approximately halve the results. The increasing impact step 3 has on the 
result as the requirement for the predictions gets stricter, is because step 3 
can only move the endpoints 6 positions and 6 positions don't have much impact 
if a prediction only has to have $10\%$ overlap but have a very large impact 
if the endpoint difference has to be below 2. 
Looking closer at the prediction from step 3 used to adjust the endpoints, it 
seems to give the same probability for every segment. The probabilities only
differs in the fifth decimal point and looks more like rounding inaccuracy
than predictions. The model does not seem to be able to learn any meaningful
thing from the input and have minimized the loss by always predicting the 
distribution of the labels in the training data. 
This failure to learn the patterns in the input data to accurately predict
the start end end of \glspl{tmh} could be because patterns is too complex or 
indistinct to learn with the size of the dataset or with the layout of the 
model.

\begin{table}
	\centering 
	\begin{tabular}{l|c|c} 
		Model & Precision & Recall \\ \hline
		Step1 & $64 \pm 3$ & $66 \pm 4$ \\ 
		Step2 & $67 \pm 3$ & $69 \pm 3$ \\ 
		Step3 & $49 \pm 3$ & $49 \pm 4$ \\
		HMM   & $52 \pm 0$ & $29 \pm 0$ \\
		TMSEG\cite{tmseg} & $87 \pm 3$ & $84 \pm 3$
	\end{tabular}
    \caption{Precision and recall for 50\% overlap and endpoints difference below 5.}
	\label{tab:pr50}
\end{table}

In Table \ref{tab:pr50} is the precision and recall from using the same 
measurement as used by TMSEG. It also shows how the model compares with 
other models. TMSEG is a lot better at both precision and recall, 
but both step 1 and step 2 is a big improvement to the simple \gls{hmm}
based model. 

\subsection{Running time}

% Step 1 796s
% Step 3 73s

\subsection{Discussion}
% What have we achieved? 
% Further work for it to be even better. 

%\subsection{Learning what the Model has Learned}
% Can we learn something from what the model has learn. By somehow visualising the weight matrices we can
% se what the model has found to be importen and sometimes something interresting aboubt the problem can be learned

\subsection{Further Work}
