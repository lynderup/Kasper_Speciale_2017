\section{Evaluation}

\subsection{Results}

\begin{table}
	\centering 
	\begin{subtable}[]{\textwidth}
		\begin{tabular}{l|c|c|c|c|c|c|c|c} 
			Model & \multicolumn{2}{|c}{10\%}& \multicolumn{2}{|c}{25\%}& \multicolumn{2}{|c}{50\%}& \multicolumn{2}{|c}{75\%}\\ 
			& Precis & Recall & Precis & Recall & Precis & Recall & Precis & Recall \\ \hline 
			Step1 & $ 82 \pm 2 $ & $ 90 \pm 2 $ & $ 80 \pm 2 $ & $ 89 \pm 1 $ & $ 72 \pm 5 $ & $ 80 \pm 6 $ & $ 50 \pm 2 $ & $ 56 \pm 3 $ \\ 
			Step2 & $ 90 \pm 2 $ & $ 92 \pm 2 $ & $ 88 \pm 2 $ & $ 90 \pm 1 $ & $ 80 \pm 6 $ & $ 82 \pm 5 $ & $ 55 \pm 4 $ & $ 57 \pm 4 $ \\ 
		\end{tabular}
		\caption{Precision and recall for 10\%, 25\%, 50\% and 75\% overlap of true \gls{tmh} and predicted \gls{tmh}.}
		\label{tab:overlap}
	\end{subtable}
	
	\begin{subtable}[]{\textwidth}
		\begin{tabular}{l|c|c|c|c|c|c|c|c} 
			Model & \multicolumn{2}{|c}{Below 10}& \multicolumn{2}{|c}{Below 5}& \multicolumn{2}{|c}{Below 2}& \multicolumn{2}{|c}{Equal 0}\\ 
			& Precis & Recall & Precis & Recall & Precis & Recall & Precis & Recall \\ \hline 
			Step1 & $ 72 \pm 2 $ & $ 79 \pm 2 $ & $ 53 \pm 4 $ & $ 59 \pm 4 $ & $ 21 \pm 2 $ & $ 23 \pm 1 $ & $  1 \pm 1 $ & $  1 \pm 1 $ \\ 
			Step2 & $ 79 \pm 3 $ & $ 81 \pm 1 $ & $ 58 \pm 6 $ & $ 59 \pm 5 $ & $ 23 \pm 2 $ & $ 24 \pm 1 $ & $  1 \pm 1 $ & $  1 \pm 1 $ \\ 
		\end{tabular}
		\caption{Precision and recall for endpoints difference between true 
			\gls{tmh} and predicted \gls{tmh} below or equal to 10, 5, 2 and 0.}
		\label{tab:endpoint}
	\end{subtable}
	\caption{Results of different overlap ratios and endpoints differences comparing the different layers of the model.
	Error rates is specified as the standard deviation of the multiple runs of the experiments.}
	\label{tab:step_compare}
\end{table}

Table \ref{tab:step_compare} contains the result for different measurements 
comparing the steps in the model. Unsurprising goes both the precision and the 
recall down when the requirement for the predictions gets stronger. In 
Table \ref{tab:overlap} can it be seen that the difference between $10\%$ and $25\%$
overlap is very small and within one standard deviation from each other, 
this is implying to that if the model have made a prediction it has more than 
$25\%$ or less than $10\%$ overlap with a true \gls{tmh}.
Table \ref{tab:endpoint} shows that the changing the endpoint requirement a little 
gives a big difference in the result, from a precision on $79 \pm 3$ for an endpoints 
difference up to $10$ to a precision on $1 \pm 1$ for no endpoints difference.
Both precision and recall is very low if the requirement for a predicted \gls{tmh}
to be correct is that there is no difference in the endpoints, ie. the prediction 
is $100\%$ correct, which means there are almost no $100\%$ correct predictions.
From both measurement types can it be seen that step 2 helps a lot on the precision.

\begin{table}
	\centering 
	\begin{tabular}{l|c|c} 
		Model & Precision & Recall \\ \hline 
		Step1 & $52.2 \pm 7.2$ & $58.6 \pm 6.7$ \\ 
		Step2 & $60.8 \pm 8.2$ & $60.9 \pm 7.1$ \\ 
		TMSEG\cite{tmseg} & $87 \pm 3$ & $84 \pm 3$
	\end{tabular}
    \caption{Precision and recall for 50\% overlap and endpoints difference below 5.}
	\label{tab:pr50}
\end{table}

Table \ref{tab:pr50} compares the model with other models.

\subsection{Discussion}
% What have we achieved? 
% Further work for it to be even better. 

\subsection{Learning what the Model has Learned}
% Can we learn something from what the model has learn. By somehow visualising the weight matrices we can
% se what the model has found to be importen and sometimes something interresting aboubt the problem can be learned

\subsection{Further Work}
